\begin{center}\huge{\textbf{Dedicatoria}}\end{center}

Dedico este proyecto a las siguientes personas:
\blinditemize
\cleardoublepage

\begin{center}\huge{\textbf{Reconocimientos}}\end{center}
\blindtext
\cleardoublepage

\begin{center}\huge{\textbf{Resumen}}\end{center}

%--------------------------------------------------------------------

El presente proyecto investiga sobre la teoría de criptografía y como la misma es implementada en la computación para el resguardo de datos, específicamente en hardware haciendo uso de FPGAs y del lenguaje de descripción de hardware Verilog. 

Como punto de partida se elegirán dos algoritmos criptográficos comúnmente empleados, para llevar a cabo un análisis comparativo de una serie de parámetros que son relevantes para su implementación en una plataforma de FPGA. Estos parámetros consideran tres de las mayores limitantes para implementación de algoritmos en FPGA, como son el consumo de celdas, la cantidad de memoria interna utilizada y finalmente el uso de celdas especializadas de aritmética o DSP.
