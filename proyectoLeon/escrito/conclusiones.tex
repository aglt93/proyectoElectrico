\chapter{Conclusiones y recomendaciones}
\section*{Conclusiones}

\begin{itemize}
\item Fue posible implementar 2 algoritmos criptográficos en un FPGA haciendo uso del lenguaje de descripción de hardware Verilog.

\item De los análisis individuales podemos concluir que:
\begin{itemize}
\item El consumo de \textit{slice L}, \textit{slice M}, \textit{slice registers} y \textit{LUTs as logic} crece exponencialmente conforme se varía el tamaño de palabra.
\item En el caso del algoritmo RC5 se puede concluir que la variación del tamaño de llave no afecta el consumo de recursos del FPGA y que la variación en la frecuencia máxima tiene un comportamiento exponencial inverso.
\item En el caso del algoritmo TEA la frecuencia máxima tiene un comportamiento lineal decreciente.
\item Idealmente el algoritmo TEA debe ser implementado para tamaños de palabra menores o iguales a 32 bits, realizando un análisis entre cuanta seguridad y velocidad se desea.
\end{itemize}


\item Del análisis comparativo se puede concluir que de ambos algoritmos, el que consume la menor cantidad de recursos y presenta la mayor frecuencia de reloj es el TEA con 32 bits de tamaño de palabra y 32 rondas. Esto usando como referencia los artículos que describen ambos algoritmos para elegir el punto comparativo del algoritmo DES.
\end{itemize}

\section*{Recomendaciones}

\begin{itemize}
\item Agregar a las métricas de análisis, el tiempo de procesamiento en ciclos de reloj de cada una de las implementaciones realizadas.
\item Mejora el diseño mediante el uso de \textit{pipeline}, esto mejorará el desempeño de los algoritmos y se podría estudiar como este cambio impacta en las métricas analizadas por este trabajo.
\item Se prodría profundizar más en el desarrollo y diseño con FPGAs para habilitar el uso de bloques de DSP del mismo y así mejorar el desempeño y consumo de los algoritmos. Esto para explotar las capacidades de un FPGA.
\item Realizar un estudio previo sobre criptografía y más que todo sobre criptoanálisis para tener un mejor criterio sobre como afectan las variaciones del tamaño de palabra, número de rondas y el tamaño de llave en la seguridad del algoritmo.
\end{itemize}
