\chapter{Introducción}
\section{Justificación}
El cifrado de datos para aplicaciones de seguridad es de vital importancia en la actualidad. Un ejemplo de esto es la fuga de datos de Home Depot en el año 2014, donde se perdió una computadora que contenía información personal de $10000$ empleados y clientes causando un costo de millones de dólares a la empresa \citep{homeDepot}. Otro ejemplo fue la noticia en mayo del 2015 sobre el buscador web \textit{UC Browser} donde quedó en evidencia que es posible robar datos personales por falta de cifrado en la transmisión de datos \citep{ucBrowser}. También en febrero del 2015 se dio a conocer la fuga de decenas de millones de datos de la compañía de seguros \textit{Anthem}. Los datos robados (números de seguro, lugar de residencia, números de teléfono, entre otros), se encontraban en una base de datos en la cual los datos se encontraban sin cifrar como indica \cite{anthem}:
\begin{center}
    \begin{minipage}{0.9\linewidth}
        \vspace{5pt}%margen superior de minipage
        {\small
            \emph{``\textit{Anthem} descubrió que \textit{hackers} irrumpieron en la base de datos y se hicieron de la información de decenas de millones de consumidores, siendo esta la mayor violación de seguridad informática en las compañías de servicios de salud. Debido a que los datos no se encontraban cifrados, eran fácilmente leíbles por los \textit{hackers}.''}
        }
 		\vspace{5pt}%margen inferior de la minipage
    \end{minipage}
\end{center}
Dada la importancia del cifrado de datos, ahora nos debemos plantear ¿Porqué es importante cifrar datos utilizando hardware?.

En el caso del cifrado implementado en software, trae consigo problemáticas como lo son la necesidad de actualizaciones y el impacto en el desempeño del computador. Este último aspecto se debe a la necesidad de dedicar recursos del sistema para cifrar y descifrar la información \citep{apricorn}.

En el caso de hardware se tienen los beneficios de que es muy confiable, rápido y conveniente. A diferencia del cifrado en software se tiene una parte de hardware dedicada al proceso de cifrado/descifrado y por tanto el desempeño de la computadora no se ve tan afectado \citep{apricorn}. Otra de las grandes ventajas del cifrado basado en hardware es la facilidad de configuración, ya que gran parte de esto proceso es eliminado debido a que el hardware se encarga directamente de esto \citep{driveTrust}.

Debido a lo expuesto anteriormente se buscó trabajar en algoritmos criptográficos en hardware ya que es un área de trabajo que se puede explotar para investigar y mejorar.

Se eligió como plataforma de trabajo un FPGA debido al bajo costo que permite usarla como un plataforma de desarrollo conveniente, la popularidad que ha alcanzado en los últimos años y las mejoras que se le han realizado para que estos dispositivos logren desempeños similares a los ASICs.

\section{Alcances y limitaciones del proyecto}
Se implementarán dos algoritmos criptográficos en un FPGA haciendo uso de Verilog como lenguaje de descripción de hardware. 

Posteriomente y mediante las herramientas de síntesis de Xilinx se realizará un análisis de métricas críticas en el desarrollo de aplicaciones en FPGAs como los son la cantidad de compuertas o celdas, el consumo de memoria interna del FPGA así como la cantidad de bloques aritméticos o de DSP que son usados por cada algoritmo. Uno de los factores que aunque tiene importancia no se tomará en cuenta es el consumo de potencia.

Inicialmente se va llevará a cabo un análisis individual de cada algoritmo, haciendo uso de las métricas anteriormente descritas y variando parámetros comunes de los algoritmos criptográficos como lo son el tamaño de la llave y la cantidad de rondas de cifrado (este último en algoritmos de tipo Feistel). 

Como segunda parte del proyecto se realizará un análisis comparativo entre ambos algoritmos eligiendo parámetros fijos para ambos.

Los análisis individuales y comparativos anteriomente mencionados no abarcarán ningún tipo de criptoanálisis de algún algoritmo con respecto otro ni de cuál sería la escogencia de los parámetros ideal para realizar un análisis comparativo entre ambos. El enfoque más bien será que a partir del análisis individual realizado se va a efectuar una escogencia de los parámetros de ambos algoritmos para realizar su comparación implementando las métricas descritas.

Para la escogencia de estos dos algoritmos se realizará a partir de una identificación de las principales ramas del cifrado para así elegir los dos algoritmos de dos de estas ramas abarcando de esta manera un tema más amplio para el análisis y discusión.

La contribución de este trabajo se limitará a realizar una escogencia de esta manera donde también tendrá mucho peso que los algoritmos que sean implementables, de manera relativamente sencilla, en un FPGA conociendo desde un principio las limitantes estáticas del hardware.

\section{Objetivos}

\subsection{Objetivo General}

Implementar dos algoritmos de cifrado en un FPGA y realizar un análisis comparativo de la implementación de ambos algoritmos, empleando una serie de parámetros previamente seleccionados.


\subsection{Objetivos Específicos}

\begin{enumerate}

\item Implementar dos algoritmos criptográficos comúnmente empleados en un FPGA utilizando el lenguaje de descripción de hardware Verilog.

\item Realizar un análisis individual para cada uno de los algoritmos, variando alguno de sus parámetros (por ejemplo el tamaño de la llave) para comparar haciendo uso de métricas como la cantidad de compuertas, el consumo de memoria interna del FPGA así como la cantidad de bloques aritméticos o de DSP que se van a ir utilizando conforme se varíe el parámetro del algoritmo elegido.

\item Realizar un análisis comparativo de los dos algoritmos implementados, utilizando como métricas la cantidad de compuertas o celdas, el consumo de memoria interna del FPGA así como la cantidad de bloques aritméticos o de DSP que son usados por cada algoritmo.


\end{enumerate}

\section{Metodología}

La metodología que se siguió para la realización del proyecto es la siguiente:

\begin{enumerate}

\item Estudios bibliográficos de:

\begin{itemize}
	\item Criptografía: importancia y como la misma se implementa en la computación.
	\item Algoritmos criptográficos: Identificación de ramas y subramas.
	\item Código e implementación de algoritmos criptográficos en diferentes lenguajes de programación.
\end{itemize}

\item Escogencia de los algoritmos criptográficos a implementar.

\item Implementación de los algoritmos criptográficos en un FPGA de Xilinx.

\item Realización del análisis individual de cada uno de los algoritmos.

\item Realización del análisis comparativo entre ambos algoritmos.

\item Realización de las conclusiones y Recomendaciones.

\end{enumerate}

\section{Desarrollo}

El presente informe se estructura para el lector de la siguiente manera:

\begin{enumerate}
\item Capítulo I: Introducción.

\item Capítulo II: Antecedentes y Marco Teórico. 

\item Capítulo III: Implementación de los algoritmos criptográficos en el FPGA.

\item Capítulo IV: Resultados de los análisis individuales y comparativos de los 2 algoritmos implementados en el FPGA.

\item Capítulo V: Conclusiones y recomendaciones.

\end{enumerate}
